\documentclass[12pt,a4paper]{article}

% Packages
\usepackage[utf8]{inputenc}
\usepackage{amsmath, amssymb}
\usepackage{graphicx}
\usepackage{hyperref}
\usepackage{geometry}
\usepackage{titlesec}
\usepackage{cite}

% Page layout
\geometry{margin=1in}

% Title formatting
\titleformat{\section}{\Large\bfseries}{\thesection.}{0.5em}{}
\titleformat{\subsection}{\large\bfseries}{\thesubsection.}{0.5em}{}

% Title information
\title{\textbf{Online Payments Banking Fraud Detection using Machine Learning}}
\author{
    Kaushiki Mondal \\
    \textit{Kalinga Institute of Industrial Technology,Bhubaneswar } \\
    \texttt{2205815@kiit.ac.in}
}
\date{\today}

% Document
\begin{document}

% Title page
\maketitle

% Abstract
\begin{abstract}
The rapid growth of online transactions has increased the risk of fraudulent activities, necessitating robust systems for detecting and preventing financial fraud. This research presents a comprehensive approach to online payment fraud detection using machine learning techniques. The study leverages a Kaggle dataset containing transaction data, preprocessing it to handle imbalances and extract relevant features. A Random Forest Model, selected for its ability to handle complex data structures and non-linear relationships, was trained and evaluated for fraud detection accuracy.

Key steps included dataset handling, preprocessing, model training , and evaluation, with metrics such as precision, recall, F1-score, and ROC-AUC guiding performance assessment. Visualization techniques provided insights into model behavior, including feature importance and prediction trends. The project culminated in a user-friendly fraud detection application built using Streamlit, offering functionalities for single and batch predictions, result visualization, and history tracking.

The results demonstrate the model's efficacy in identifying fraudulent transactions with high accuracy while ensuring interpretability. The application's design emphasizes usability and real-time prediction, making it practical for deployment in real-world scenarios. This research underscores the potential of machine learning in enhancing the security of online payment systems and paves the way for further advancements in fraud detection methodologies.
\end{abstract}



\end{document}

